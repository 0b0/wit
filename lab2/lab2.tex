
%%% Local Variables: 
%%% mode: latex
%%% TeX-master: t
%%% End: 
\documentclass[12pt,letterpaper]{article}

\usepackage[polish]{babel}
\usepackage[utf8]{inputenc}
\usepackage{polski}
\usepackage[T1]{fontenc}
\frenchspacing
\usepackage{indentfirst}

\usepackage{amsmath} % just math
%\usepackage{amssymb} % allow blackboard bold (aka N,R,Q sets)
\usepackage{ulem}
\linespread{1.6}  % double spaces lines
\usepackage[left=1in,top=1in,right=1in,bottom=1in,nohead]{geometry}
\begin{document}

\linespread{1} % single spaces lines
\small \normalsize %% dumb, but have to do this for the prev to work
\begin{flushright}
Laboratorium programowania w języku Java -- 2 
\footnote{Na podstawie kursów MIT OCW} \\
Piotr Kowalski,
\today
\end{flushright}

\section{Projekt}
Utwórz projekt Shapes w środowisku Netbeans z plikami, które znajdują
się w dołączonym archiwum {\it shapes.zip}. Wszystkie klasy powinny
znajdować się w pakiecie \verb+shapes+.

\section{Klasy}
Analiza, testowanie i modyfkacja klas specyfikujących klasę \verb+Shape+ (Kształt).

Zaimplementowano dwie klasy \verb+Square+ i \verb+Circle+. Dla każdej z tych klas jesteśmy zainteresowani otrzymaniem pola powierzchni figury (zauważ, że metoda \verb+area+ jest dziedziczona z interfejsu \verb+Shape+).

\begin{enumerate}
\item Przetestuj \verb+Square+ oraz \verb+Circle+ - z wykorzystaniem JUnit 3.
\item Zaimplementuj klasę \verb+Triangle+ (trójkąt), która również będzie dziedziczyła z \verb+Shape+. Klasa \verb+Triangle+ powinna mieć ten sam zestaw metod co pozostałe klasy dziedziczące. Klasa powinna być inicjalizowana wartościami długości podstawy oraz wysokości na nią opuszczonej (\verb+base+ oraz \verb+height+).  
\item Pamiętaj o testowaniu każdej metody.
\end{enumerate}

\section{Kolekcje}
Być może wygodniejszą formą pracy z kształtami będzie zgromadzenie ich w zbiorze \verb+ShapeSet+. Klasa ShapeSet zarządza zbiorem obiektów typu Shape. 

\begin{enumerate}
\item Zastanów się jak chcesz przechowywać kształty? Co będzie odpowiednie \verb+set+, \verb+list+, \verb+queue+ czy może \verb+map+. Jakie są zalety i wady?
\item Utwórz metodę \verb+addShape+ dodającą kształt do zbioru. Zapewnij, że żadne dwa te same kształty nie znajdą się w zbiorze.
\item Zaimplementuj metodę \verb+iterator+ zwracającą iterator do zbioru elementów.
\item Uzupełnij metodę \verb+toString+.
\item Pamiętaj o testowaniu każdej metody.
\end{enumerate}

\section{Wyszukiwanie i plik}
Teraz gdy mamy implementację klas reprezentujących kształty oraz zbior kształtów należy zaimplementować dwie metody w klasie \verb+Shapes+ (nie pomylić z klasą \verb+Shape+). 

\begin{enumerate}
\item \verb+findLargest+ - nalezy znaleźć figury o największym polu i utworzyć listę a nastepnie przekazać ją jako wartość zwracaną.
\item \verb+readShapesFromFile+ - Aby ułatwić sobie tworzenie obiektów do projektu dołączony jest plik z deklaracją kształtów. Należy go wczytać i utworzyć obiekt ShapeSet zawierający kształty występujące w pliku. (rozwiązanie należy wyszukać w dokumentacji lub przykładach w internecie --- IO Javy zostanie dokładniej omówione na następnych zajęciach)
\item Pamiętaj o testowaniu każdej metody.
\end{enumerate}

\end{document}
